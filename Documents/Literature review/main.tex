\documentclass[review]{cmpreport}
\usepackage{natbib}


\title{Third Year Project: Literature Review}
\author{Matthew Taylor}
\registration{100151729}
\supervisor{Dr Rudy Lapeer}
\ccode{CMP-6013Y}

\begin{document}

\maketitle

\begin{section}{Introduction}

\subsection{Aim of the project}
The project aims to implement a first person shooter game which uses procedural generation to provide a novel and interesting gaming experience. There any a lot of ways procedural content generation (PCG) can be applied to games, so this literature review will explore the broad themes of the field to find specific areas to focus on.

\subsection{Areas of knowledge required}
There are many PCG techniques, which have been developed and used in video games since the 1970s. I will need to gain a strong understanding of this field in order to find interesting areas to focus on. \par
I will also need to learn the Unity platform, in order to implement my ideas. 

\subsection{Report structure}
The report will provide a summary of the PCG field, identifying key themes and the current state of the field, including interesting "unsolved problems" or under-explored areas. \par
I will then focus on a few promising areas to explore the literature in detail, with a view to informing my project's aim and objectives.

\end{section}

\begin{section}{Key themes}
\subsection{Types of Procedural Content Generation}
There are two main types of procedural content generation, which use similar methods but achieve different goals. They are \textbf{endless content generation} and \textbf{efficient content storage}.

\subsubsection{Efficient content storage}
Prevalent in the early days of PCG, efficient content storage exploits procedural content generation to create large amounts of data without needing to store it - it can simply be re-generated using a deterministic algorithm every time it is required. An example of this is the 1984 game \textit{Elite}, as documented in \citet{spufford_2003}. Here, the game developers discuss using a pseudorandom number generator with pre-defined seeds to generate large amounts of content, without needing to store any of the content itself. \par
A similar technique is adopted for image generation, as per \cite{Perlin:1985:IS:325165.325247}. Perlin describes using well-defined stochasic functions that can be parametised to produce realistic textures procedurally. This means that textures can be generated when required or when a program is initialised, rather than needing to store them. \par
A lot has changed in computers since the 80s - memory and storage constraints are now much less of an issue, so using PCG to efficiently store large amounts of content has been used less.

\subsubsection{Endless content generation}
As storage constraints have been eclipsed by the rapid development of technology, the generation of content has become more of a concern. Increasing sophistication of games and the expectations of players mean content creation is more time-consuming and costly. This has resulted in an increase in the use of PCG to generate content. \par
This has become a wide field, resulting in the creation of a taxonomy of PCG in \cite{Hendrikx:2013:PCG:2422956.2422957}. It describes six layers of PCG in games:
\begin{enumerate}
    \item Game bits (eg. textures, buildings)
    \item Game space (eg. maps, lakes)
    \item Game systems (eg. roads)
    \item Game scenarios (eg. puzzles, stories)
    \item Game design (eg. rules)
    \item Derived content (eg. leaderboards)
\end{enumerate}

\end{section}

\begin{section}{Key themes}
In chapter 6 of \cite{Hendrikx:2013:PCG:2422956.2422957}, the paper discusses "recommendations for future research". One of the recommendations details the generation of realistic indoor game spaces. I will explore this field and the literature surrounding it in more detail. 

\subsubsection{Room generation}
\cite{taylor-parberry} goes into detail about "computerised clutter" and the challenges involved in making a procedurally generated room look realistic. They list five properties that are desirable in a room generator: novelty, structure, interest, speed and controllability. The paper describes in detail how to procedurally generate "clutter" in a room which does not look computer generated, by using the concept of "anchor points". These points are adjusted with gaussian noise to provide natural looking variation around points and patterns that can be designer-adjusted, making them look human-constructed when in reality they are generated by computers.\par
I will incorporate some of these techniques into the decoration of my room generator.

\subsubsection{Layout generation}
\cite{Hendrikx:2013:PCG:2422956.2422957} in Chapter 3 discuss common methods of PCG. In relation to how indoor spaces are generated, they discuss how common methods are Pseudo-random Number Generators (PRNG) and Generative Grammars (GG). \par
These methods are well suited to producing feasible indoor spaces required in games. Although \cite{Hendrikx:2013:PCG:2422956.2422957} detail other techniques, like genetic algorithms and fractal spatial generation, these are less common in games. I suspect this is because the output of layouts they produce, despite being procedurally generated, will either follow a discernable pattern (in the case of fractal algorithms) or will be "too random" (in the case of genetic algorithms) to appeal to human players. My aim is to create a procedural generator that feels realistic in order to be compelling and interesting to the player, so I feel these approaches may not be suitable.

\subsubsection{Path generation}
TBC

\end{section}




\begin{section}{Evaluation and analysis}
I use several kinds of sources in my literature review, from academic journals to books and articles about techniques used. Here, I will evaluate and analyse my sources.

\subsection{\cite{spufford_2003}}
This is a book that directly interviews and quotes the designers of the game Elite. Despite the age of the source (2003) it describes techniques used in the 80s - so it's age is acceptable. The fact it directly quoted developers about the algorithms they used makes this a reliable source.


\subsection{\cite{Perlin:1985:IS:325165.325247}}
This Perlin paper is quite old, so there are certain techniques that have superceded some of the world. Nevertheless, this was a very influential paper (cited 427 times on acm.org) and the work presented is still very relevant today. Perlin Noise is still a commonly used algorithm in procedural generation, especially in things like realistic-looking heightmaps in modern tools.

\subsection{\cite{Hendrikx:2013:PCG:2422956.2422957}}
This paper is relatively recent (2013) which makes the summary it provides still accurate to this day. It is cited (36 times on acm.org) by others in the industry. I have done some comparison against http://pcg.wikidot.com/ which maintains a list of games using PCG, which confirms there have been no large advances or practical examples of PCG in games since this was published. 

\subsection{\cite{taylor-parberry}}
This paper is very specific although not well cited (1 citation on researchgate.net). It is also not very recent, published in 2010. These facts mean I am not going to read too much into the paper, especially as they primarily present only one technique of generating "computerised clutter". However, their subjective summary of what makes procedural generation of clutter in a room look convincing is useful and compelling enough that I have included it. 

\end{section}


\bibliography{sources}

\end{document}
